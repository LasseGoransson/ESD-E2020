
\documentclass{article}
\usepackage{commonnote} % own package in ~/texmf

\begin{document}
\title{Lab 1 - Embed}
\date{}
\maketitle

\section{Actors}

\begin{itemize}
    \item Customer
    \item Cashier
    \item Manager
\end{itemize}

\section{Use cases}


\section{Requirements}
\begin{table}[h]
    \centering
    \begin{tabular}{l|l|l|l|l}
        Id& Description  & Evaluation  & Analysis & Test\\
        \hrule
        & Able to register items & Input test  && \\
        & Optional receipt for customer & Input test && \\
        & Cashier needs to immediately verify eligibility after first restricted item  & Functional inspection&&\\
        & Cashier must be able to modify registered item and quantity & Input test&& \\
        & Query Item IDs, Prices from central box  & Functional inspection&& \\
        & Communication with central box using Ethernet zeroMQ & Integration test && \\
        & Customer Display - Item scanned now, quantity, price total & Functional test&& \\
        & Cashier Display  - Line items, Total Cost, Transaction ID & Functional test&&\\ 
        & Scalable up to 50 stations &Design inspection&& \\
        & System can handle payment using credit card & Input test&& \\
        & System can handle payment using cash & Input test&& \\
        & Cashier can cancel transaction & Functional test && \\
        & Customer is able to return items & Functional test&& \\

    \end{tabular}
    \caption{caption}
    \label{tab:label}
\end{table}

\section*{Lecture 6 - Requirements as Presented in the Class}%
\label{sec:lecture_6_requirements_as_presented_in_the_class}

\subsection*{Requirements as shown to the class}%

\begin{enumerate}
	\item Scan 2 items per second.
	\item Scaleable up to 50 stations.
	\item Pay by cash or card.
	\item Transaction cancellation.
	\item zmq use for communication (port and ip). No the exact server, merely the interface.
	\item Info for customer on customer display.
		\begin{itemize}
			\item Total at the end.
			\item Current total.
			\item Current item.
		\end{itemize}
	\item Cash tracking (number fo sales).
	\item optional receipt printing
		\begin{itemize}
			\item A simple item is a item representation with only the most necessary information. (id,name,barcode).
			\item We should be able to make a new receipt ( \texttt{ newReceipt() } ).
			\item Complete transaction by printing the receipt \texttt{ completeTransaction() }.
			\item We need to be able to access the receipt in different contexts \texttt{getRecipt()}.
			\item \texttt{getPaymentStatus()} show us the status, either it is \texttt{payed()} or \texttt{canceled()}, these returns strings verifying the
				information from the database.
		\end{itemize}
	\item Use data base interface.
	\item Employee screen. 
		\begin{itemize}
			\item Active receipt.
			\item Current/final total.
		\end{itemize}
	\item Possibility of returning item/items. Policy/Protocol:
		\begin{itemize}
			\item 30 day limitation.
			\item Requires receipt.
			\item Time stamp in receipt.
		\end{itemize}
	\item Make it user friendly (When you can scan one item, a multiplier can be added
		in order to not scan i.e 100 bananas). We should add shortcuts for tedious 
		tasks in the system.
	\item Edit receipt.
	\item USER FRIENDLY!
	\item Special item alert - Notifying when special items are being scanned by the system.
	\item Acceptable wrong info: Name, price. We assume that the barcode always is correct. The barcode is ground truth.
	\item Query item data. Meaning the customer wants the price of some product, maybe it is not shown on the shelf.
	\item Luha algorithm for credit card validation.
	\item Reboot station functionality for simple maintenance, done by the employee.
	\item Any receipt size, which regards to the number of items in a transaction - a must.
	\item USB for barcode scanner, card swiper, customer display.
	\item A visual representation of the system in form of diagrams.
\end{enumerate}

Amount of stock is not up for us.
\\ \\

Return item Protocol:
\begin{enumerate}
	\item Customer provides receipt
	\item The employee should input the receipt id. 
	\item The system checks the id to verify the time stamp etc. This is done on the
		station box.
	\item How to represent the returning phenomenon in the system,
		a negative buy? Return items class as mentioned by Aske.
\end{enumerate}

\section*{Lecture 7}%
\label{sec:lecture_7}
\subsection*{Project}%
\label{sub:project}

\begin{itemize}
	\item Groups of 6
	\item Separate the tasks among the members
	\item Individual projects
	\item Hand in a video of your individual work (A video showing all the functionalities of the station).
	\item Optional video as a group.
\end{itemize}



\end{document}

